\addcontentsline{toc}{chapter}{Introduzione}
\chapter*{Introduzione}
	Il mondo moderno è ormai pervaso dalla \emph{crittografia}. Quotidianamente e spesso inconsapevolmente utilizziamo funzioni crittografiche per le normali operazioni della vita quotidiana. Controllare il conto sull'home-banking, scambiarsi messaggi tramite servizi di messaggistica o anche navigare in internet utilizzando il protocollo \acs{HTTPS} sono azioni che svolgiamo ormai con naturalezza. I sistemi moderni rendono trasparente all'utente l'utilizzo di tali tecnologie ma ciò non vuol dire che non ci siano.
	
	Una \emph{funzione crittografica}\index{Funzione crittografica} è un oggetto matematico astratto che trasforma con l'utilizzo di una chiave, un dato in input (plaintext) in una sua rappresentazione diversa (ciphertext) il più possibile non riconducibile al dato originale. Questa funzione deve poi essere implementata in un programma che girerà su un \emph{dispositivo crittografico} in un certo ambiente, presentando perciò caratteristiche fisiche peculiari. Esempi di \dispp potrebbero essere smartcard, chiavette \acs{USB}, chip dedicati montati su dispositivi general purpose (smartphone, notebook) o periferiche progettate e costruite apposta per effettuare questo unico compito.
	
	In passato si guardava ad un \disps semplicemente come una black-box che riceveva un plaintext e restituiva un ciphertext (encryption) e viceversa (decryption). Gli attacchi erano basati sulla conoscenza del ciphertext (ciphertext-only attacks) o di alcune coppie di entrambi (known plaintext attacks). Con l'accesso al meccanismo di encryption o di decryption, anche solo temporaneo, si possono attuare anche altri due tipi di attacchi (rispettivamente chosen-plaintext e chosen-ciphertext)\cite{dispenseCS}.
	
	Al giorno d'oggi si è consapevoli del fatto che un \disps ha spesso altri input oltre al plaintext e altri output oltre al ciphertext. Gli input differenti dal plaintext possono essere interazioni col mondo esterno come modifiche al voltaggio della corrente, condizioni atmosferiche particolari o sollecitazioni fisiche. Il nostro interesse sarà però focalizzato sulle informazioni (facilmente misurabili) che vengono lasciate trapelare dai dispositivi stessi oltre al ciphertext come ad esempio il tempo di esecuzione di un programma, le radiazioni emesse, suoni, luci e quant'altro chiamate \emph{side-channel informations}\index{Side-channel informations}.
	 
	 Il resto della tesi è organizzata nel seguente modo. Nel capitolo 1 verrà definita una classificazione dei side-channel attacks e verrà presentata una panoramica dello stato dell'arte. Il capitolo 2 approfondirà i \emph{cache attacks} e in particolar modo quelli basati sul tempo.
	 Nel capitolo 3 verrà presentato approfonditamente il recente attacco \emph{SPECTRE} che ha afflitto tutti i recenti processori AMD, ARM e Intel.
	 Nel capitolo 4 verrà affrontato il problema della generalizzazione di questi attacchi utilizzando la \ac{QIF} una nuova tecnica che permette di stabilire proprietà di confidenzialità delle informazioni.
