\chapter{Introduzione}
	Una \emph{funzione crittografica} è un oggetto matematico astratto che trasforma con l'utilizzo di una chiave, un dato in input (plaintext) in una sua rappresentazione diversa (ciphertext) il più possibile non riconducibile al dato originale. Questa funzione deve poi essere implementata in un programma che girerà su un \emph{dispositivo crittografico} in un certo ambiente, presentando perciò caratteristiche fisiche peculiari. Esempi di \dispp potrebbero essere smartcard, chiavette USB, chip dedicati montati su dispositivi general purpose (smartphone, notebook) o periferiche progettate e costruite apposta per effettuare questo unico compito.
	
	In passato si guardava ad un \disps semplicemente come una black-box che riceveva un plaintext e restituiva un ciphertext (encryption) e viceversa (decryption). Gli attacchi erano basati sulla conoscenza del ciphertext (ciphertext-only attacks) o di alcune coppie di entrambi (known plaintext attacks). Con l'accesso al meccanismo di encryption o di decryption, anche solo temporaneo, si possono attuare anche altri due tipi di attacchi (rispettivamente chosen-plaintext e chosen-ciphertext)\cite{dispenseCS}.
	
	Al giorno d'oggi si è consapevoli del fatto che un \disps ha spesso altri input oltre al plaintext e altri output oltre al ciphertext. Gli input differenti dal plaintext possono essere interazioni col mondo esterno come modifiche al voltaggio della corrente, condizioni atmosferiche particolari o sollecitazioni fisiche. L'interesse sarà però focalizzato sulle informazioni (facilmente misurabili) che vengono lasciate trapelare dai dispositivi stessi oltre al ciphertext come ad esempio il tempo di esecuzione di un programma, le radiazioni emesse, suoni, luci e quant'altro.
	
	 I \emph{side-channel attacks} sono metodi di criptanalisi che sfruttano questo tipo di informazioni insieme ad altre tecniche di analisi per recuperare la chiave utilizzata dal dispositivo\cite{standaert2010introduction}.
	 
	 Il resto della tesi è organizzata nel seguente modo. Nel capitolo 1 definirò una classificazione dei side-channel attacks e presenterò una panoramica dello stato dell'arte. Il capitolo 2 approfondirà i \emph{cache attacks} e in particolar modo quelli basati sul tempo.
	 Nel capitolo 3 presenterò approfonditamente il recente attacco \emph{SPECTRE} che ha afflitto tutti i recenti processori AMD, ARM e Intel.
	 Nel capitolo 4 affronterò il problema della generalizzazione di questi attacchi utilizzando la Quantitative information-flow analysis (QIF) una nuova tecnica che permette di stabilire proprietà di confidenzialità delle informazioni.
