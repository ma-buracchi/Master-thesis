%--------------------------------------------------------------
% thesis.tex 
%--------------------------------------------------------------
% - template for the main file of Informatica@Unifi Thesis 
% - based on Classic Thesis Style Copyright (C) 2008 
%   Andr\'e Miede http://www.miede.de   
%--------------------------------------------------------------
\documentclass[openright,titlepage,oneside,fleqn,
	headinclude,12pt,a4paper,footinclude,makeidx]{scrbook}%twoside
%--------------------------------------------------------------
\usepackage[italian]{babel}
\usepackage[utf8]{inputenc} 
\usepackage[T1]{fontenc} 
\usepackage[square,numbers]{natbib} 
\usepackage[fleqn]{amsmath}  
\usepackage{ellipsis}
\usepackage{listings}
\usepackage{subfig}
\usepackage{caption}
\usepackage{appendix}
\usepackage{siunitx}
\usepackage{lipsum}
\usepackage{dia-classicthesis-ldpkg}
\usepackage[eulerchapternumbers,linedheaders,subfig,beramono,eulermath,
parts,dottedtoc]{classicthesis}
\usepackage[italian,noabbrev]{cleveref}
\usepackage{imakeidx}
%---------------------------------------------------------------
\newcommand{\myItalianTitle}{Titolo italiano\xspace}
\newcommand{\myEnglishTitle}{Titolo inglese\xspace}
\newcommand{\myDegree}{Corso di Laurea Magistrale in Informatica\xspace}
\newcommand{\myCurriculum}{Resilient and secure cyberphysical systems\xspace}
\newcommand{\myName}{Marco Buracchi\xspace}
\newcommand{\myProf}{Michele Boreale\xspace}
\newcommand{\myFaculty}{Scuola di Scienze Matematiche, Fisiche e Naturali\xspace}
\newcommand{\myUni}{\protect{Università degli Studi di Firenze}\xspace}
\newcommand{\myLocation}{Firenze\xspace}
\newcommand{\myTime}{Anno Accademico 2017-2018\xspace}
\newcommand{\mycopyright}{\includegraphics[width=1.5cm]{logo/cc.png} 
	\href{https://creativecommons.org/licenses/by-nc-sa/4.0/}{Creative
		Commons Attribution-NonCommercial-ShareAlike 4.0 International (CC BY-NC-SA 4.0)  }\xspace}
\newcommand{\disps}{dispositivo crittografico }
\newcommand{\dispp}{dispositivi crittografici }
\newcommand{\myfloatalign}{\centering} 
%--------------------------------------------------------------
\newlength{\abcd} % for ab..z string length calculation
% how all the floats will be aligned
\setlength{\extrarowheight}{3pt} % increase table row height
\captionsetup{format=hang,font=small}
%--------------------------------------------------------------
% Layout setting
%--------------------------------------------------------------
\graphicspath{{img/}}
\usepackage{geometry}
\geometry{
	a4paper,
	ignoremp,
	bindingoffset = 1cm, 
	textwidth     = 13.5cm,
	textheight    = 21.5cm,
	lmargin       = 3.5cm, % left margin
	tmargin       = 4cm    % top margin 
}

\lstset{ %
	% backgroundcolor=\color{back},  	% choose the background color; you must add \usepackage{color} or \usepackage{xcolor}; should come as last argument
	basicstyle=\footnotesize,        		% the size of the fonts that are used for the code
	breakatwhitespace=false,        % sets if automatic breaks should only happen at whitespace
	breaklines=true,                % sets automatic line breaking
	captionpos=b,                   % sets the caption-position to bottom
	commentstyle=\color{gray},   	% comment style
	frame=trLB,	                % adds a frame around the code
	frameround=fttt					% round corner (use f instead t to edge corner)
	keepspaces=true,                % keeps spaces in text, useful for keeping indentation of code (possibly needs columns=flexible)
	keywordstyle=\color{blue},      % keyword style
	%language=Python,               	% the language of the code
	numbers=none,                   % where to put the line-numbers; possible values are (none, left, right)
	numbersep=5pt,                  % how far the line-numbers are from the code
	numberstyle=\tiny\color{black}, % the style that is used for the line-numbers
	rulecolor=\color{black},        % if not set, the frame-color may be changed on line-breaks within not-black text (e.g. comments (green here))
	showspaces=false,               % show spaces everywhere adding particular underscores; it overrides 'showstringspaces'
	showstringspaces=false,         % underline spaces within strings only
	showtabs=false,                 % show tabs within strings adding particular underscores
	stepnumber=1,                   % the step between two line-numbers. If it's 1, each line will be numbered
	stringstyle=\color{OrangeRed},  % string literal style
	tabsize=2                   	% sets default tabsize to 2 spaces
}

\makeindex[intoc]
%--------------------------------------------------------------
\begin{document}
\frenchspacing
\raggedbottom
\pagenumbering{roman}
\pagestyle{plain}
%--------------------------------------------------------------
% Frontmatter
%--------------------------------------------------------------
\include{titlePage}
\pagestyle{scrheadings}
%--------------------------------------------------------------
% Mainmatter
%--------------------------------------------------------------
\pagenumbering{arabic}
\tableofcontents
\listoffigures
\listoftables
\cleardoublepage
\thispagestyle{empty}
\begin{flushright}
\null\vspace{\stretch {1}}
\emph{"Da campo a campo, nel tetro grembo della notte,\\s’avverte appena il brusio di entrambe le armate,\\sicché le sentinelle appostate quasi possono udire\\i mormorii furtivi delle sentinelle nemiche" \break --- Enrico V, William Shakespear} \vspace{\stretch{2}}\null
\end{flushright}
\cleardoublepage
%----------------CAPITOLI--------------------------------------
\chapter{Introduzione}
	I \emph{side-channel attacks} sono metodi di criptanalisi che sfruttano informazioni estrapolate dall'ambiente fisico nel quale lavorano dei dispositivi crittografici\cite{canteaut2006understanding}.
	
	prova di visualizzazione delle citazioni successive \cite{kocher2018spectre}.

\chapter{Side-channel attacks}

	I \emph{side-channel attacks} sono metodi di criptanalisi che sfruttano le side-channels informations insieme ad altre tecniche di analisi per recuperare la chiave utilizzata da un \disps\cite{standaert2010introduction}.\begin{figure}
		
		\begin{center}
			\includegraphics[scale=.6]{sideChannelLeakage}
			\caption{Esempio di side-channel attack}
			\label{fig:attack}
		\end{center}
	\end{figure}
	
	Nella \cref{fig:attack} si può vedere una configurazione tipo di side-channel attack. Da una parte c'è il dispositivo che implementa la funzione crittografica e accanto c'è lo strumento utilizzato per rilevare le grandezze fisiche prodotte dal dispositivo attaccato. La cosa fondamentale è che questo tipo di attacchi non vanno a colpire direttamente la funzione crittografica ma sfruttano le informazioni fisiche dell'ambiente intorno al dispositivo.
	
	L'analisi di questi metodi ha acquisito notevole interesse dato che questo tipo di attacchi possono essere montati velocemente e molto spesso non richiedono hardware particolare e costoso. Con pochi euro si possono ad esempio acquistare in comuni negozi di bricolage o elettronica apparecchi in grado di analizzare il consumo elettrico di un dispositivo. Con tali apparecchi è possibile montare in pochi secondi un attacco di tipo \emph{Simple Power Analysis}\cite{mangard2002simple} che verrà spiegato più avanti. 
	
	Il governo degli USA, nel suo "Orange book"\cite{latham1986department} indica dei requisiti di sicurezza per i sistemi operativi. Questo documento introduce i primi standard per l'\emph{information leakage}. Purtroppo la letteratura specializzata è però molto variegata e disomogenea quindi, come prima cosa, cerchiamo di trovare un modo per classificare i vari tipi di attacchi in maniera tale da avere una visione più sistemistica del settore.
	
	\section{Background}
	
		In questa sezione verranno stabiliti dei parametri per classificare i vari tipi di attacchi side-channel.
	
		\subsection{Tipi di canali}
	
			Nel lavoro di \emph{Ge, Yarom, Cock e Heiser}\cite{ge2016survey} vengono fornite alcune definizioni che utilizzeremo nel prosieguo di questa tesi. La prima distinzione che è necessario fare è quella tra side-channel e \emph{covert-channel}. Con i primi ci si riferisce ai canali che lasciano \emph{accidentalmente} filtrare informazioni sensibili (ad esempio una chiave crittografica) in una comunicazione tra due partecipanti fidati. I secondi sono quelli creati e sfruttati dall'attaccante ad esempio tramite l'utilizzo di Trojan e che \emph{deliberatamente} lasciano filtrare le informazioni. In questo lavoro verranno trattati solamente i primi.
			
			L'altra differenza fondamentale per quello che riguarda i canali è quella tra canali di tipo \emph{storage} e canali di tipo \emph{timing}. I canali di tipo storage vengono sfruttati per ottenere qualcosa di direttamente visibile nel sistema (valore dei registri, valore di ritorno di una system call, ecc.). Quelli di tipo timing vengono sfruttati andando ad osservare variazioni del tempo di esecuzione di un programma (o di parti di esso).
			
		\subsection{Tipi di attacco}
		
			\emph{Standaert} nel suo lavoro \cite{standaert2010introduction} utilizza altre due dimensioni interessanti per classificare questi attacchi; l'\emph{invasività} e l'\emph{attività}. 
			
			Si definisce invasivo un attacco che richiede un disassemblamento del dispositivo attaccato per avere accesso diretto ai suoi componenti interni (wiretapping o sensori collegati direttamente all'hardware). Un attacco non invasivo, al contrario, sfrutta solamente le informazioni disponibili esternamente (quasi sempre involontarie) come il tempo d'esecuzione o l'energia consumata.
			
			Si definisce attivo un attacco che cerca di interferire con il corretto funzionamento del dispositivo (fault-injection) mentre un attacco passivo si limita ad osservare il comportamento del dispositivo durante il suo lavoro senza disturbarlo. 
			
		\subsection{Grandezza fisica osservata}
		
			Una caratteristica principale di questi attacchi è sicuramente la grandezza fisica che viene osservata per montare l'attacco. Teoricamente, qualunque grandezza fisica misurabile può essere sfruttata ma alcune si prestano maggiormente rispetto ad altre. 
			
			Il tempo e il consumo energetico sono le più sfruttate ma non sono di certo le uniche. \emph{Genkin, Shamir e Tromer} nel loro lavoro \cite{genkin2014rsa} vanno ad ascoltare i rumori prodotti dal processore. \emph{Ferrigno e Hlavac}\cite{ferrigno2008aes} osservano la luce (qualche fotone) emessa dai transistor nel passaggio di stato da 0 a 1. \emph{Martinasek, Zeman e Trasy} sfruttano i campi elettromagnetici creati dai chip. \emph{Giraud}\cite{giraud2004dfa} sfrutta la tecnica della fault-injection e analizza i risultati delle computazioni.
			
			Questo elenco assolutamente non esaustivo delle tecniche utilizzate può far capire quanto variegato ed eterogeneo (nonché in continua evoluzione) sia questo settore.
			
			Tutti questi attacchi (e anche altri) verranno approfonditi più avanti.
			
		\subsection{Hardware attaccato}
		
			Gli attacchi possono essere suddivisi anche in base alla componente hardware che viene attaccata. Anche in questo caso ci sono componenti più attaccati di altri (cache e processori) ma non mancano esempi di attacchi a monitor\cite{van1985electromagnetic}, tastiere\cite{asonov2004keyboard} o stampanti\cite{backes2010acoustic}.
			
		\subsection{Algoritmo attaccato}
		
			Un'ultima classificazione può essere effettuata andando a discriminare gli attacchi secondo l'algoritmo crittografico attaccato. In questo caso i due maggiori algoritmi attaccati sono senza dubbio AES ed RSA.
			
			\begin{table}[]
				\centering
				\caption{Classificazione dei principali attacchi analizzati}
				\label{tab:attacchi}
				\begin{tabular}{c|c|c|c|c|c} \hline
					Articolo 					& Grandezza                    & Hardware   & Algoritmo & Invasivo & Attivo \\ \hline
					\cite{genkin2014rsa}		& Suono                        & CPU        & RSA       & No       & No     \\ \hline
					\cite{ferrigno2008aes}		& Luce                         & Transistor & AES       & Sì       & No     \\ \hline
					\cite{kocher2018spectre}	& Tempo                        & Cache      & RSA       & No       & No     \\ \hline
					\cite{mangard2002simple}	& Consumo elettrico            & Smartcard  & AES       & No       & No     \\ \hline
					\cite{martinasek2012simple}	& Campo elettromagnetico       & Chip       & AES       & No       & No     \\ \hline
					\cite{giraud2004dfa}		& Risultato della computazione & Smartcard  & AES       & Sì       & Sì    
				\end{tabular}
			\end{table}
		
			Nella \cref{tab:attacchi} si è provato a classificare con i criteri sopra definiti i 6 attacchi che verranno approfonditi in questo lavoro.
			
	\section{Attacchi basati sul suono}
	\lipsum[1-5]
	\section{Attacchi basati sulla luce}
	\lipsum[1-5]
	\section{Attacchi basati sul tempo}
	\lipsum[1-5]
	\section{Attacchi basati sul consumo elettrico}
	\lipsum[1-5]
	\section{Attacchi basati sul campo elettromagnetico}
	\lipsum[1-5]
	\section{Attacchi basati sui risultati delle computazioni}
	\lipsum[1-5]
\chapter{Timing attacks}
	\index{Timing attacks}Come anticipato nel capitolo precedente, approfondiremo adesso la tipologia di attacchi basati sul tempo focalizzandoci maggiormente su quelli che hanno come obiettivo la cache del processore. 
	
	L'idea di base che sta sotto i timing attacks è quella che l'esecuzione di un determinato programma, al variare delle operazioni che vengono eseguite e al variare degli input, impiega tempi diversi per portare a termine il proprio compito.
	
	\begin{figure}
		\begin{center}
			\lstinputlisting[language=Java]{code/timingBase.txt}
			\caption{Esempio di una funzione attaccabile tramite un timing attack}
			\label{fig:timingBase}
		\end{center}
	\end{figure}
	
	Ad esempio il codice in \cref{fig:timingBase} se fatto girare con la stringa ("passwordToBeStolen") impiegherà un tempo maggiore rispetto allo stesso programma fatto girare con la stringa ("foo"). Nel primo caso infatti verrà scansionata tutta la stringa mentre nel secondo caso si interromperà immediatamente. Questa informazione può essere utilizzata dall'attaccante per capire la stringa esatta. Per portare questo concetto al livello che ci interessa vediamo prima delle nozioni fondamentali sulla cache del processore. 
	
	\section{La cache del processore}
		Dato che la differenza di velocità tra le memorie e la capacità di calcolo dei processori aumenta sempre di più \cite{hennessy2011computer} la banda del bus di comunicazione e la velocità di accesso alla memoria principale sono diventati un fattore limitante sul throughput generale del processore. Questo collo di bottiglia viene attenuato dall'utilizzo delle cache\index{Cache}. 
		
		La cache è infatti un piccolo banco di memoria molto veloce sito all'interno di ogni core che il processore utilizza per immagazzinare i valori delle celle di memoria accedute più recentemente. 
		
		\subsection{Struttura della cache}
			I processori moderni hanno generalmente due livelli di cache per ogni core (L1 e L2). Considerando che l'accesso alla memoria principale in media impiega dai 50 ai 150 \emph{ns} mentre l'accesso alla cache L1 utilizza un tempo nell'ordine degli 0.3 \emph{ns} si può capire l'enorme differenza di prestazioni che possono essere raggiunte utilizzando questo tipo di memoria.
		
			Nella \cref{fig:cachei5} si può vedere l'architettura del processore quadcore Intel Core i5-3470. La gerarchia delle cache è organizzata in una memoria L1 di 64KB (divisa in 32KB per le istruzioni e 32KB per i dati) ed una memoria L2 da 256KB per ogni core ed un terzo livello chiamato L3 o \ac{LLC} da 6MB comune a tutti e 4 i core.
			
			\begin{figure}
				\begin{center}
					\includegraphics[scale=0.6]{cachei5}
					\caption{Architettura del processore Intel Core i5-3470}
					\label{fig:cachei5}
				\end{center}
			\end{figure}
			
			Andiamo ad analizzare più nel dettaglio le caratteristiche di una singola cache\cite{ge2016survey,yarom2014flush+}.
			
			\subsubsection{Cache lines}
				\index{Cache lines}Per sfruttare la località spaziale le caches sono divise in lines. Una cache line contiene un blocco di bytes adiacenti (generalmente di dimensione congrua ad una potenza di 2) caricati dalla memoria. Se uno qualunque dei bytes deve essere rimosso (si parla di \emph{evicting}) per far spazio ad un altro dato, tutta la line viene ricaricata.
				
			\subsubsection{Associatività}
				Teoricamente una qualunque posizione di memoria può essere mappata in una qualunque cache line ed una cache ad \emph{n} lines potrebbe contenere \emph{n} linee qualunque dalla memoria. Questo tipo di cache viene chiamato \emph{fully-associative cache}\index{Fully-associative cache} ed è la migliore in teoria perché può sempre essere usata al massimo delle sue capacità e i cache miss si hanno solamente quando non c'è più spazio libero nella cache. In pratica però questo si traduce in un controllo in parallelo di tutte le linee che aumenta la complessità architetturale e il consumo di energia.
				
				L'estremo opposto è chiamato \emph{direct-mapped cache}\index{Direct-mapped cache}. In questo sistema ogni locazione di memoria può stare in una sola cache line, ben determinata da una funzione di indicizzazione. Due locazioni di memoria che mappano sulla stessa cache line non possono essere immagazzinate contemporaneamente e il loading di una comporta inevitabilmente l'evicting dell'altra. Questo potrebbe portare ad avere dei miss anche con la cache semivuota.
				
				Concretamente viene utilizzata una via di mezzo tra queste due soluzioni chiamata \emph{set-associative cache}\index{Set-associative cache}. La cache viene divisa in \emph{sets} (generalmente di dimensione compresa tra 2 e 24 lines) in cui ogni indirizzo viene controllato in parallelo come in una fully-associative cache. In quale set viene mappato un blocco di memoria viene calcolato come per una direct-mapped cache da una funzione del suo indirizzo. Una cache con \emph{n} line sets viene chiamata \emph{n-way associative}.
				
				Si può notare che le direct-mapped e le fully-associative cache non sono altro che casi particolari di set-associative cache rispettivamente 1-way associative ed N-way associative (dove N è il numero di linee della cache).
				
			\subsubsection{Inclusività}
				Una caratteristica che verrà sfruttata per montare l'attacco è l'\emph{inclusività}\index{Inclusività}. 
				
				Ogni livello superiore di cache contiene un sottoinsieme dei dati contenuti dal livello direttamente inferiore. Per mantenere questa caratteristica, quando viene eseguito un evicting di un dato da un livello inferiore, questo viene rimosso anche da tutti i livelli superiori.
				
	\section{Cache attacks}
		Per capire come funzionano la maggior parte degli attacchi alle cache prendiamo in considerazione un array di dati. Quando un elemento di questo array viene acceduto possono verificarsi una di queste due condizioni:
		
		\begin{enumerate}
			\item Il dato è presente in cache, si verifica una hit e viene recuperato molto velocemente.
			\item Il dato non è presente in cache, si verifica una miss e bisogna aspettare che venga recuperato dalla memoria principale.
		\end{enumerate}
		
		La differenza tra le due esecuzioni è notevole (diversi ordini di grandezza) ed è questa l'informazione utilizzata nell'attacco.
		
		\subsection{Tassonomia}
			Una prima classificazione dei cache attacks si basa sullo stato della cache al momento dell'attacco\cite{canteaut2006understanding}.
			
			\begin{itemize}
				\item \emph{Empty initial state}\index{Empty initial state} (reset attacks): questi attacchi si basano sull'assunzione che nessun dato che dovrà essere utilizzato dalla vittima è presente in cache.
				\item \emph{Forged initial state}\index{Forged initial state} (initialization attacks): in questo caso l'attaccante deve essere in grado di portare la cache in uno stato noto prima di poter effettuare l'attacco.
				\item \emph{Loaded initial state}\index{Loaded initial state} (micro-architecture attacks): la cache contiene tutti i dati necessari alla vittima per eseguire il programma.
			\end{itemize}
			
			In \cite{lipp2016armageddon,ge2016survey} si classificano gli attacchi in base all'approccio utilizzato:
			
			\begin{itemize}
				\item \emph{Evict+Time}\cite{osvik2006cache}\index{Evict+Time}: Questo attacco è di tipo loaded initial state e suppone che tutti i dati che servono alla vittima siano già in cache. Questa condizione può essere ottenuta facendo eseguire una prima volta la funzione vittima. Con questa base, l'attaccante fa eseguire la funzione alla vittima calcolandone il tempo di esecuzione. Successivamente esegue una evict di una cache line caricando in memoria un dato che va a sovrascrivere uno presente in cache e fa eseguire nuovamente la funzione vittima. Se il tempo di questa ultima esecuzione è maggiore del precedente vuol dire che la funzione ha cercato di utilizzare il dato che è stato rimosso dalla cache ed ha dovuto aspettare di recuperarlo dalla memoria principale.
				\item \emph{Prime+Probe}\cite{osvik2006cache}\index{Prime+Probe}: Questo è un attacco di tipo forged initial state. L'attaccante precarica uno o più set della cache con dati propri. Dopo l'esecuzione della funzione vittima prova a riaccedere ad i propri dati. Se la funzione vittima non ha utilizzato lines mappate nei cache set occupati dall'attaccante, egli otterrà solo cache hit. Al contrario, se c'è stato l'evict di qualche line allora capirà quale ha utilizzato la vittima.
				\item \emph{Flush+Reload}\cite{yarom2014flush+}\index{Flush+Reload}: Questo attacco è una variante di Prime+Probe. L'attacco si divide in tre fasi. Nella prima fase l'attaccante esegue l'evict della linea a cui è interessato utilizzando l'istruzione \emph{clflush} che invalida il dato su tutti i livelli della cache. Nella seconda fase aspetta che la vittima esegua la propria funzione. Nella terza fase l'attaccante ricarica la linea che aveva rimosso. Se la risposta è veloce vuol dire che la vittima l'ha portata in cache durante l'esecuzione della sua funzione.
				\item \emph{Evict+Reload}\cite{gruss2015cache}\index{Evict+Reload}: Una variante del Flush+Reload che utilizza la eviction al posto dell'istruzione di flush.
				\item \emph{Flush+Flush}\cite{gruss2016flush+}\index{Flush+Flush}: Diversamente da tutti i precedenti approcci, in questo caso non si esegue nessun accesso alla memoria ma l'attaccante si basa solamente sul tempo impiegato dall'istruzione \emph{clflush}. In \cite{lipp2016armageddon} si fa vedere come l'esecuzione di questa funzione abbia tempi differenti se chiamata su un indirizzo presente in cache o meno.  
			\end{itemize}
		
		\section{Contromisure possibili}
			Le difese da questo tipo di attacchi sono sia software che hardware e si dividono in 6 grandi famiglie\cite{ge2016survey}
			
			\begin{description}
				\item[Tecniche a tempo costante:] L'idea di base è quella di rendere il comportamento del codice che esegue operazioni critiche indipendente dai dati. Per esempio cercare di rendere una funzione crittografica indipendente sia dalla chiave che dall'input. Questo può essere ottenuto facendo eseguire istruzioni inutili per uniformare il tempo di esecuzione o accedendo a dati casuali dalla memoria per confondere l'attaccante sull'utilizzo della cache. Queste soluzioni ovviamente portano ad una drastica perdita di prestazioni. Il tempo di esecuzione dovrà infatti tendere al tempo di esecuzione massimo ogni volta che sarà necessario richiamare la funzione.
				\item[Inserimento di rumore:] Questa famiglia di contromisure tende a rendere inutilizzabili le misure ottenute dall'attaccante inserendo in ogni evento osservabile da qualsiasi processo una quantità di rumore tale da renderne impossibile una qualunque analisi\cite{hu1992reducing}.
				\item[Imporre determinismo:] In questo caso si cerca di eliminare qualsiasi tipo di misura sul tempo eliminando completamente le variazioni di tempo visibile. Ad esempio in \cite{aviram2012efficient} si propone di eliminare completamente l'accesso al tempo reale fornendo all'esterno solamente un clock virtuale il cui avanzamento è completamente deterministico e indipendente dalle azioni di componenti vulnerabili. Per ottenere questo risultato si cerca di sincronizzare tutti i clock con l'esecuzione di un singolo processo che esegue in tempo costante.
				\item[Partizionare il tempo:] In questo caso si cerca di suddividere il tempo in sezioni nelle quali si fornisce un accesso esclusivo all'hardware condiviso. Ci sono diverse tecniche per ottenere questo risultato uno dei quali è la cancellazione completa della cache ad ogni context switch(\emph{cache flushing}\index{Cache flushing}). Questo ovviamente porta ad una perdita in prestazioni molto grande e si è passati al \emph{lattice scheduling}\index{Lattice scheduling} che esegue il flushing della cache non ad ogni context switch ma solo nel passaggio da processi sensibili a processi inaffidabili. Attacchi di tipo Prime+Probe hanno bisogno di analizzare molto spesso lo stato della cache della vittima. Questa cosa viene evitata imponendo un tempo minimo di esecuzione per le componenti vulnerabili entro il quale non possono essere prelazionate.
				 
			\end{description}
\chapter{SPECTRE attacks}
	\index{Spectre}In questo capitolo verrà presentato \emph{SPECTRE}\cite{kocher2018spectre}, un tipo di attacco molto recente che sfrutta una vulnerabilità presente nella maggior parte dei processori moderni (Intel, AMD e ARM) e per il quale, al momento, non esistono contromisure.
	
	La vulnerabilità che viene sfruttata da questo tipo di attacco è la cosiddetta \emph{esecuzione speculativa}.
	
	\section{Esecuzione speculativa}
		\index{Esecuzione speculativa}L'esecuzione speculativa è una tecnica utilizzata dai processori per migliorare le prestazioni che consiste nel cercare di "indovinare" il risultato di un branch per eseguire preventivamente alcune istruzioni.
		
		Supponiamo ad esempio che l'esecuzione del programma dipenda da un controllo su di un valore non presente in cache che quindi deve essere recuperato dalla memoria principale. Questo può portare ad un attesa di svariate centinaia di cicli clock prima che questo valore sia disponibile. Invece di aspettare tutto questo tempo inutilmente, il processore cerca di indovinare il risultato del controllo, salva lo stato attuale dei suoi registri, e procede ad eseguire speculativamente il ramo del branch che ritiene più plausibile (supponiamo il ramo then). Quando poi arriverà il valore effettivo dalla memoria, il controllo verrà effettivamente effettuato. Se il risultato è quello aspettato (true nel nostro caso), si procede con la computazione e saranno stati risparmiati tutti quei cicli di clock che sarebbero stati persi nell'attesa. Se la scelta si rivela sbagliata (false), il processore scarta tutti i risultati dell'esecuzione speculativa, si riporta allo stato che si era salvato precedentemente ed esegue l'altro ramo del branch (else).
		
		Questa ottimizzazione sembra perfetta in quanto in caso di successo, si risparmiano molti cicli di clock mentre in caso di insuccesso il risultato è paragonabile a quello che avremmo ottenuto aspettando il dato senza eseguire alcuna istruzione.
		
		Il responsabile di questa scelta è una piccola unità all'interno del processore chiamata \ac{BP}.
		
		\subsection{Branch predictor}
			\index{Branch Predictor}Esistono svariati tipi di branch predictor; andiamo a vedere come funziona uno dei più semplici, il \emph{one-level branch predictor}\index{One-level branch predictor} a 2 bit.
			
			\begin{figure}
				\begin{center}
					\includegraphics[scale=.35]{bp2bit}
					\caption{Automa di predizione di un one-level branch predictor a 2 bit}
					\label{fig:bp2bits}
				\end{center}
			\end{figure}
		
			Come da schema in \cref{fig:bp2bits} un one-level branch predictor può essere descritto con un semplice automa a 4 stati.
			
			\begin{enumerate}
				\item \emph{Strongly not taken:} in questo stato il \ac{BP} sceglierà il ramo else del branch. In caso di risultato effettivamente negativo resterà in questo stato altrimenti passerà allo stato 2.
				\item \emph{Weakly not taken:} in questo stato il \ac{BP} ha già osservato una esecuzione then ma la sua scelta resterà ancora il ramo else. Se il controllo si rivelerà false, il \ac{BP} tornerà allo stato 1 ma se si rivelerà true andrà allo stato 3 dal quale inizierà a scegliere il ramo then.
				\item \emph{Weakly taken:} come detto in precedenza, in questo stato il \ac{BP} inizierà ad eseguire speculativamente il ramo then. Se da questo stato si ottiene un false, torneremo allo stato 2, altrimenti passeremo al 4.
				\item \emph{Strongly taken:} questo stato è il duale dello stato 1. In questa situazione il \ac{BP} eseguirà il ramo then rimanendo in questo stato se otterrà un true e tornando allo stato 3 se otterrà un false (continuando comunque ad eseguire il ramo then).
			\end{enumerate}
		
			In questo caso vediamo come l'esecuzione consecutiva di al più due rami then ci porta sicuramente in uno stato in cui il ramo scelto dal \ac{BP} sarà sicuramente quello then. Questa informazione sarà molto utile quando dovremo effettuare un training sul \ac{BP} per convincerlo ad eseguire il ramo then quando si troverà davanti ad un certo branch.
			
	\section{L'attacco}
		L'attacco SPECTRE induce la vittima ad eseguire speculativamente operazioni che non dovrebbero essere eseguite durante l'esecuzione corretta del programma. Da tali operazioni si otterranno poi le informazioni ricercate tramite un side-channel temporale.
		
		L'attacco si può scomporre in tre fasi:
		
		\begin{enumerate}
			\item \emph{Fase di setup:} in questa fase l'attaccante esegue delle operazioni che convincono il \ac{BP} ad eseguire il ramo then in caso si rendesse necessaria una esecuzione speculativa. In questa fase si cerca anche di costruire tale necessità ad esempio eseguendo letture di memoria che rimuovono dalla cache un valore che sarà poi necessario successivamente. Come ultima cosa l'attaccante può iniziare a preparare la porzione di cache dalla quale estrarrà il valore che vuole carpire alla vittima (ad esempio eseguendo il flush o l'evict di una line o di un set).
			\item \emph{Esecuzione speculativa:} in questa fase il processore esegue speculativamente delle istruzioni che esporrano informazioni confidenziali della vittima recuperabili tramite un side-channel. Tale esecuzione può esporre dati sensibili attraverso una vasta gamma di side-channels ma nell'articolo gli autori si concentrano sulla possibilità di recuperare un valore che risiede ad un indirizzo preciso nella memoria della vittima attraverso un attacco di tipo Flush+Reload o Evict+Reload.
			\item \emph{Recupero del dato:} come ultimo passo, viene montato l'attacco alla cache (Flush+Reload o Evict+Reload). Il recupero del dato si ottiene andando a misurare il tempo necessario alla lettura dall'indirizzo di memoria presente nella line sotto attacco.
		\end{enumerate}
	
		Vediamo adesso un esempio pratico di attacco.
		
		\subsection{Esempio}
		
			Consideriamo il caso di una funzione che contiene al suo interno il seguente codice che riceve un intero x da una fonte non fidata:
			
			\begin{lstlisting}[language={C}, frame={none},basicstyle={\footnotesize}]
				if (x < array1_size) {
					y = array2[array1[x] * 256];
				}
			\end{lstlisting}
			
			Il processo che esegue il codice ha accesso ad un array di bytes \emph{array1} di dimensione \emph{array1\_size} ed un secondo array, \emph{array2}, di dimensione pari a 64KB.
			
			La funzione inizia con un controllo su x, necessario per essere sicuri di non permettere la lettura di porzioni di memoria al di fuori di array1. Durante l'esecuzione speculativa di questo codice però, il \ac{BP} può selezionare il ramo then relativo a questo controllo ad esempio nel seguente caso:
			
			\begin{itemize}
				\item il valore di x viene scelto in maniera malevola in maniera tale da far puntare array1[x] ad un byte segreto \emph{k} che risiede da qualche parte nella memoria della vittima.
				\item \emph{array1\_size} e \emph{array2} non sono presenti nella cache ma \emph{k} lo è.
				\item operazioni precedenti hanno restituito un valore di x corretto, addestrando il \ac{BP} a scegliere il ramo then.
			\end{itemize}
		
			Questa situazione può presentarsi in maniera normale o può essere creata dall'attaccante ad esempio leggendo grandi quantità di memoria per riempire la cache di valori completamente scorrelati ed effettuando una chiamata legittima ad una funzione che utilizzi \emph{k}.
			
			A questo punto, quando il programma inizia a girare il processore esegue il confronto tra x e \emph{array1\_size}. La lettura di \emph{array1\_size} si traduce in un cache miss ed il processore richiede il dato dalla memoria principale. Durante l'attesa il \ac{BP} assume che il risultato dell' \emph{if} sarà \emph{true}, eseguirà speculativamente la somma di x all'indirizzo base di \emph{array1} e richiederà il dato presente all'indirizzo appena calcolato. Questa operazione si tradurrà in una cache hit e verrà restituito molto velocemente il valore del byte segreto \emph{k}. L'esecuzione speculativa continua il suo percorso e verrà calcolato l'indirizzo di \emph{array2[k*256]}. La richiesta del dato contenuto a questo indirizzo si tradurrà in una cache miss e verrà richiesta una lettura dalla memoria principale. Durante questa seconda attesa, al processore arriva finalmente il valore di \emph{array1\_size}. Dopo aver eseguito il confronto il processore si accorge che l'esecuzione speculativa era errata ed esegue un rollback allo stato precedente al branch. Il problema sorge in questo momento. La richiesta di lettura rimasta sospesa viene comunque portata a termine e il valore relativo non viene rimosso dalla cache.
			
			Per completare l'attacco l'attaccante non deve fare altro che rilevare questo cambiamento nello stato della cache per recuperare il byte segreto \emph{k}.
			
			Nel caso più semplice, quello cioè in cui l'attaccante ha accesso diretto ad \emph{array2}, egli non dovrà fare altro che provare  a leggere tutte le posizioni di \emph{array2[n*256]} per tutti i valori di $n \in (0\dots255)$. Solamente uno di questi sarà presente in cache (quello per \emph{n} = \emph{k}) e verrà restituito velocemente mentre tutti gli altri verranno restituiti in tempi molto lunghi. Prendendo l'unico valore restituito in tempo breve, l'attaccante avrà trovato il byte segreto \emph{k}.
			
\chapter{Proof of concept}
	\lipsum
%-----------------BIBLIOGRAFIA---------------------------------
\bibliography{bib/bibliografia}
\bibliographystyle{unsrt}
%--------------------------------------------------------------
\addcontentsline{toc}{chapter}{Acronimi}
\chapter*{Acronimi}
\begin{acronym}[CAGD]
	\acro{CNES}{Centre National d’Etudes Spatiales}
	\acro{DPA}{Differential Power Analysis}
	\acro{HTTPS}{HyperText Transfer Protocol over Secure Socket Layer}
	\acro{LED}{Light Emitting Diode}
	\acro{PICA}{Picosecond Imaging Circuit Analysis}
	\acro{SPA}{Simple Power Analysis}
  	\acro{USB}{Universal Serial Bus}
\end{acronym}
\printindex
\end{document}
%--------------------------------------------------------------
