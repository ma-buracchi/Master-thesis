\chapter{Timing attacks}
	\index{Timing attacks}Come anticipato nel capitolo precedente, approfondiremo adesso la tipologia di attacchi basati sul tempo focalizzandoci maggiormente su quelli che hanno come obiettivo la cache del processore. 
	
	L'idea di base che sta sotto i timing attacks è quella che l'esecuzione di un determinato programma, al variare delle operazioni che vengono eseguite e al variare degli input, impiega tempi diversi per portare a termine il proprio compito.
	
	\begin{figure}
		\begin{center}
			\lstinputlisting[language=Java]{code/timingBase.txt}
			\caption{Esempio di timing attack}
			\label{fig:timingBase}
		\end{center}
	\end{figure}
	
	Ad esempio il codice in \cref{fig:timingBase} se fatto girare con la stringa ("passwordDaRubare") impiegherà un tempo maggiore rispetto allo stesso programma fatto girare con la stringa ("foo"). Nel primo caso infatti verrà scansionata tutta la stringa mentre nel secondo caso si interromperà immediatamente. Questa informazione può essere utilizzata dall'attaccante per capire la stringa esatta.