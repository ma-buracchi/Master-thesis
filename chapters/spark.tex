\chapter{SPARK}
	In questo capitolo verrà analizzato \ac{SPARK}\index{SPARK}, un attacco basato sul progetto SPECTRE capace di recuperare chiavi (o più in generale "segreti") evitando il controllo della password.
	
	\section{Scenario}
		Immaginiamo di partire dalla funzione attaccata da SPECTRE (\cref{list:vulnerabile}). Nello scenario più semplice si può pensare che \emph{array1} (da adesso sarà chiamato \emph{secret}) contenga delle chiavi predefinite assegnate ad ogni utente che vengono utilizzate come indici  per accedere ad \emph{array2} che contiene le informazioni riservate di tutti gli utenti. Dato che queste informazioni sono riservate, il tutto è protetto con una password. 
		
		L'utente \emph{x}, identificato tramite il proprio \emph{userID}, che vuole consultare i propri dati dovrà inserire ID e password (salvate nell'array delle password \emph{passwordDigest}). In caso di controllo positivo, si utilizzerà l'indice contenuto in \emph{secret}$[$\emph{userID}$]$ per andare a recuperare l'informazione che si riferisce ad \emph{x}. In caso di controllo negativo verrà ovviamente negato l'accesso ai dati.
		
		Tale funzione potrebbe essere implementata dal \cref{list:spark}:
		\codice{28}{32}{Funzione attaccata da SPARK}{list:spark}
		
		Supponiamo adesso che l'attaccante abbia accesso ad \emph{array2} (come supposto anche dall'attacco SPECTRE) ma che non abbia accesso a \emph{secret} ed ovviamente neanche all'array delle password. L'obiettivo del nostro attacco è quello di ricavare, per un utente casuale \emph{x}, il rispettivo \emph{secret}$[$\emph{x}$]$, per poter successivamente avere accesso all'informazione riservata, il tutto senza conoscere la password.
		
		La precisione del risultato ottenuto dipende molto dal processore e dal tipo di dati utilizzato per rappresentare i vari valori. Nella nostra implementazione abbiamo utilizzato il tipo di dato \emph{int}; questo non ci permette di recuperare esattamente \emph{secret}$[$\emph{x}$]$ ma siamo in grado di localizzarlo entro un certo intervallo \emph{secret}$[$\emph{x}$] \pm \delta$ con, nel nostro setup sperimentale, $$\delta = \frac{\text{dimensione di una line}}{\text{dimensione del tipo di dato}} = \frac{512}{32} = 16$$.
		
		\begin{figure}
			\begin{center}
				\includegraphics[scale=.6]{lineSize}
				\caption{Contenuto di una cache line nel nostro setup}
				\label{fig:lineSize}
			\end{center}
		\end{figure}
		
		Come si può vedere in \cref{fig:lineSize} ogni volta che viene richiesto il dato presente in una posizione di \emph{array2} (ad esempio \emph{array2}$[$\emph{i}$]$) e questo non si trova in cache, viene caricata un'intera line che può contenere sedici posizioni. Da quel momento, quando verrà richiesta una qualunque di quelle sedici, si otterrà una hit.
		 